% $groups$delegate: false
% $groups$delegate$into: false
% $groups$matter: false
% $groups$matter$into: false

% $matter[-header,-no-header]:
% - .[no-header]

% $matter[-matter-guard,no-header]:
% - verbatim: \begingroup \let\ifsourcelinks\iftrue
%   condition: source-link
% - .[matter-guard]
% - verbatim: \endgroup % \let\ifsourcelinks
%   condition: source-link

\begingroup
\providecommand\ifsourcelinks{\iffalse}
\providecommand\url[1]{\texttt{#1}}
\providecommand\href[2]{#2}

\title{%
    Зимняя олимпиадная школа СУНЦ МГУ --- 2016
    (материалы секции математики)}
\author{%
    редактор: Ю.\,В.\,Тихонов \\
    \texttt{\href{mailto:july.tikh@gmail.com}{july.tikh@gmail.com}}}
\date{8--12 января 2016 г.}

\maketitle

\subsection*{Немного о~группах}

Занятия проходили в~трех группах: 9-1, 9-2 и~10--11.

Школьники 9 класса были разбиты на~группы по~итогам вступительной олимпиады.

\subsection*{Немного о~структуре}

Материалы разбиты по~группам, в~пределах каждой группы разбиты по~темам.

Материалы, общие для нескольких групп, дублируются (это касается только
вступительной и~заключительной олимпиад, на~самом деле).
\ifsourcelinks
Все материалы сопровождаются ссылками на~исходные файлы \LaTeX.
\fi

\endgroup % \def\ifsourcelinks \def\url

