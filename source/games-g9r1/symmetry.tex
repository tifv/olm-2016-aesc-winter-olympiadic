% $date: 2016-01-08
% $timetable:
%   g9r1:
%     2016-01-08:
%       3:

\section*{Математические игры. Симметрия}

% $authors:
% - Юлий Васильевич Тихонов

\emph{%
В~каждой из~следующих игр ходят по~очереди два игрока.
Укажите, кто из~них выиграет при правильной игре.}

\begin{problems}

\itemx{$^\circ$}
На~столе лежат две стопки монет: в~одной из~них 30 монет, а~в~другой~--- 20.
За~ход разрешается взять любое количество монет из~одной стопки.
Проигрывает тот, кто не~сможет сделать ход.

\item
У~ромашки $n$ лепестков.
За~ход разрешается сорвать либо один лепесток, либо два рядом растущих
лепестка.
Проигрывает игрок, который не~сможет сделать ход.
% второй игрок может разрезать ромашку пополам.

%\item
%Двое играют на~доске $20 \times 14$ клеток.
%Каждый по~очереди отмечает квадрат по~линиям сетки (любого возможного размера)
%и~закрашивает его.
%Выигрывает тот, кто закрасит последнюю клетку.
%Дважды закрашивать клетки нельзя.

\item
В~каждой клетке доски $11 \times 11$ стоит шашка.
За~ход разрешается снять с~доски любое количество подряд идущих шашек либо
из~одного вертикального, либо из~одного горизонтального ряда.
Выигрывает снявший последнюю шашку.

\item
Вершины правильного $n$-угольника закрашены черной и~белой краской через одну.
Двое играют в~следующую игру.
Каждый по~очереди проводит отрезок, соединяющий вершины одинакового цвета.
Эти отрезки не~должны иметь общих точек (даже концов) с~проведенными ранее.
Побеждает тот, кто сделал последний ход.
\\
\subproblem $n = 10$.
\qquad
\subproblem $n = 12$.
% (a) центральная симметрия
% (b) первый игрок может разрезать круг пополам

\item
Дан прямоугольный параллелепипед размерами
\\
\subproblem $4 \times 4 \times 4$,
\quad
\subproblem $4 \times 4 \times 3$,
\quad
\subproblem $4 \times 3 \times 3$,
\\
составленный из~единичных кубиков.
За~ход разрешается проткнуть спицей любой ряд, если в~нем есть хотя~бы один
непроткнутый кубик.
Проигрывает тот, кто не~может сделать ход.
% (a) центральная симметрия
% (b) осевая симметрия
% (c) первый игрок выкалывает центральный 4×1×1, дальше осевая симметрия

\item
Коля и~Витя играют в~следующую игру на~бесконечной клетчатой бумаге.
Начиная с~Коли, они по~очереди отмечают узлы клетчатой бумаги~--- точки
пересечения вертикальных и~горизонтальных прямых.
При этом каждый из~них своим ходом должен отметить такой узел, что после этого
все отмеченные узлы лежали в~вершинах выпуклого многоугольника
(начиная со~второго хода Коли).
Тот из~играющих, кто не~сможет сделать очередного хода, считается проигравшим.
Кто выигрывает при правильной игре?
% второй выбирает произвольный центр симметрии (не совпадающий с первой
% точкой). еще неплохо бы доказать после этого, что игра завершится

\item
Двое игроков по~очереди расставляют в~каждой из~$24$~клеток поверхности куба
$2 \times 2 \times 2 $ числа $1, 2, 3, \ldots, 24$
(каждое число можно ставить один раз).
Второй игрок хочет, чтобы суммы чисел в~клетках каждого кольца из~8 клеток,
опоясывающего куб, были одинаковыми, а~первый хочет ему помешать.
% у каждой клетки есть противоположная — та, которая «глядит» на нее из
% противоположной грани

%\item
%Двое игроков по~очереди выставляют на~доску $65 \times 65$ по~одной шашке.
%При этом ни в~одной линии (горизонтали или вертикали) не~должно быть больше
%двух шашек.
%Кто не~может сделать ход~--- проиграл.
%% http://problems.ru/view_problem_details_new.php?id=105123

\item
Петя и~Вася играют на~доске размером $7 \times 7$.
Они по~очереди ставят в~клетки доски цифры от~$1$ до~$7$ так, чтобы ни~в~одной
строке и~ни~в~одном столбце не~оказалось одинаковых цифр.
Первым ходит Петя.
Проигрывает тот, кто не~сможет сделать ход.
Кто из~них сможет выиграть, как~бы ни играл соперник?
% http://problems.ru/view_problem_details_new.php?id=64694
% это кубик! число — глубина в кубике.

\end{problems}

