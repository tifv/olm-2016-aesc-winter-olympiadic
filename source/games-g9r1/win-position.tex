% $date: 2016-01-08
% $timetable:
%   g9r1:
%     2016-01-10:
%       1:

\section*{Математические игры. Позиции}

% $authors:
% - Юлий Васильевич Тихонов

\emph{%
В~каждой из~следующих игр ходят по~очереди два игрока.
Укажите, кто из~них выиграет при правильной игре.}

\begin{problems}

\item
Шахматный король стоит в~левом нижнем углу шахматной доски.
За~один ход короля можно передвинуть на~одно поле вправо, на~одно поле вверх
или на~одно поле по~диагонали <<вправо-вверх>>.
Игрок, который поставит короля в~правый верхний угол доски,
\quad
\subproblemx{$^\circ$} выигрывает;
\quad
\subproblem проигрывает.

\item
В~левом нижнем углу клетчатого прямоугольника $15 \times 22$ стоит ладья.
Маша и~Вика по~очереди передвигают ее~на~любое количество клеток либо вправо,
либо вверх.
Первой ходит Маша.
Выигрывает та~девочка, которая поставит ладью в~правый верхний угол доски.

\item
Имеются две кучи камней, в~первой $14$ камней, во~второй~--- $21$.
Маша и~Вика по~очереди берут сколько угодно камней с~любой кучи, но только
с~одной.
Выигрывает та девочка, которая возьмет последний камень.

\item
Игра начинается с~числа $2$.
За~ход разрешается прибавить к~имеющемуся числу любое натуральное число,
меньшее его.
Выигравшим считается тот, в~результате хода которого получится $2014$.

\item
Игра начинается с~числа $60$.
За~ход разрешается уменьшить имеющееся число на~любой из~его делителей.
Проигрывает тот, кто получит ноль.

\item
Имеется две кучки спичек:
\\
\subproblem $101$ и~$201$
\quad
\subproblem $100$ и~$201$
\quad
\subproblemx{*} $24$ и~$56$
\quad
спичек.
\\
За ход разрешается уменьшить количество спичек в одной из кучек на число,
являющееся делителем количества спичек в другой кучке.
Выигрывает тот, после чьего хода спичек не остается.

\item
Имеются три кучи камней, в~первой $4$~камня, во~второй~--- $5$,
в~третьей~--- $7$.
Маша и~Вика по~очереди берут сколько угодно камней с~любой кучи, но~только
с~одной.
Выигрывает та~девочка, которая возьмет последний камень.

\item
В~левом нижнем ближнем углу клетчатого параллелепипеда $5 \times 6 \times 8$
стоит ладья.
Маша и~Вика по~очереди передвигают~ее~на~любое количество клеток либо вправо,
либо вверх, либо вглубь.
Первой ходит Маша.
Выигрывает та~девочка, которая поставит ладью в~правый верхний дальний угол
параллелепипеда.

\end{problems}

