% $date: 2016-01-11
% $timetable:
%   g1011:
%     2016-01-11:
%       3:

\section*{Перечислительные задачки}

% $authors:
% - Глеб Александрович Погудин

\begin{problems}

\item
Сколько существует четырехзначных чисел, делящихся на~$4$, в~десятичной записи
которых нет цифр $4$, $5$, $6$, $8$?

\item
Число $72350$ написали семь раз подряд, при этом получилось $35$-значное число
\[
    72350723507235072350723507235072350
\, .\]
Из~этого $35$-значного числа требуется вычеркнуть две цифры так, чтобы
полученное после вычеркивания $33$-значное число делилось на~$15$.
Сколькими способами это можно сделать?

\item
На~новогодний вечер пришли несколько супружеских пар, у~каждой из~которых было
от~$1$ до~$10$ детей.
Дед Мороз выбирал одного ребенка, одну маму и~одного папу из~трех разных семей
и~катал их в~санях.
Оказалось, что у~него было ровно $3630$ способов выбрать нужную тройку людей.
Сколько всего могло быть детей на~этом вечере?

\item
Фабрика выпускает карандаши семи цветов радуги.
Требуется составить из~этих карандашей неупорядоченный набор из~$10$ штук таким
образом, чтобы в~наборе имелось не~менее четырех красных карандашей, не~менее
двух синих и~хотя~бы один зеленый.
Сколько существует способов сделать это?

\item
Назовем \emph{рамкой} квадрат $n \times n$, из~которого удалили квадрат
$(n - 2) \times (n - 2)$ с~тем~же центром.
Клеточки рамки можно красить в~белый и~черный цвета.
Назовем раскраску рамки \emph{хорошей,} если рамку можно разрезать на~доминошки
так, что каждая доминошка состоит из~клеточек разных цветов.
Сколько существует хороших раскрасок?

\item
Клетки шахматной доски раскрашиваются в~$3$ цвета~--- белый, серый и~черный~---
таким образом, чтобы соседние клетки, имеющие общую сторону, отличались цветом,
однако резкая смена цвета (то~есть соседство белой и~черной клеток) запрещена.
Найдите число таких раскрасок шахматной доски (раскраски, совпадающие при
повороте доски на~$90^{\circ}$ и~$180^{\circ}$, считаются разными).

\item
Назовем \emph{лестницей высоты~$n$} фигуру, состоящую из~всех клеток квадрата
$n \times n$, лежащих не~выше диагонали.
Сколькими различными способами можно разбить лестницу высоты~$n$ на~несколько
прямоугольников, стороны которых идут по~линиям сетки, а~площади попарно
различны?

\end{problems}

