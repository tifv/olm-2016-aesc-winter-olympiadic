% $date: 2016-01-12
% $timetable:
%   g1011:
%     2016-01-12:
%       1:

\section*{Заключительная олимпиада}

% $authors:
% - Глеб Александрович Погудин

\subsection*{Довывод}

\begin{problems}

\item
Докажите, что среди любых $100$ трехзначных чисел найдутся четыре числа
с~одинаковой суммой цифр.

\item
Найдите количество таких упорядоченных пар чисел $(a, b)$, где
$1 \leq a, b \leq 700$, таких, что $a + b$ делится на~$7$,
а~$ab$ делится на~$5$.

\item
В~выпуклом четырехугольнике $ABCD$:
$\angle CAD + \angle BCA = 180^{\circ}$ и~$AB = BC + AD$.
Докажите, что $\angle BAC + \angle ACD = \angle CDA$.

\item
Докажите, что не~существует попарно различных натуральных чисел
$x$, $y$, $z$, $t$, для которых было~бы справедливо соотношение
\[
   x^x + y^y = z^z + t^t
\, . \]

\item
В~каждой клетке квадратной таблицы $m \times m$ клеток стоит либо натуральное
число, либо нуль.
При этом, если на~пересечении строки и~столбца стоит нуль, то~сумма чисел
в~<<кресте>>, состоящем из~этой строки и~этого столбца, не~меньше $m$.
Докажите, что сумма всех чисел в~таблице не~меньше, чем $m^2 / 2$.

\end{problems}


\subsection*{Вывод}

\begin{problems}

\item
Внутри квадрата $ABCD$ расположен квадрат $KMXY$.
Докажите, что середины отрезков $AK$, $BM$, $CX$ и~$DY$ также являются
вершинами квадрата.

\item
По~кругу стоят $10^{1000}$ натуральных чисел.
Между каждыми двумя соседними числами записали их наименьшее общее кратное.
Могут~ли эти наименьшие общие кратные образовать $10^{1000}$ последовательных
чисел (расположенных в~каком-то порядке)?

\item
$25$~дачников получили садовые участки.
Каждый участок представляет собой квадрат $1 \times 1$, и~все участки вместе
составляют квадрат $5 \times 5$.
Каждый дачник враждует не~более, чем с~тремя другими дачниками.
Докажите, что можно распределить участки таким образом, чтобы участки
враждующих дачников не~были~бы соседними (по~стороне).

\end{problems}

