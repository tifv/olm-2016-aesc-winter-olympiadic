% $date: 2016-01-08
% $timetable:
%   g1011:
%     2016-01-08:
%       1:
%       2:

\section*{Вступительная олимпиада}

% $authors:
% - Глеб Александрович Погудин

\subsection*{Довывод}

\begin{problems}

\item
Найдите все $x$, при которых $\tg x$ и~$\tg 2x$ одновременно принимают целые
значения.
% пусть tg x = a
% тогда tg 2x = 2a / (1 - a^2) = 2a / (1 - a)(1 + a)
% так как a взаимно просто с 1 - a и 1 + a, то 1 - a^2 = \pm 1, \pm 2
% такие дела

\item
Значения квадратного трехчлена $y = x^2 + ax + b$ в~двух последовательных целых
точках~--- соответственно квадраты двух последовательных натуральных чисел.
Докажите, что значения трехчлена во~всех целых точках~--- точные квадраты.
% пусть в точке x имеем n^2, в точке x + 1 имеем (n + 1)^2
% Тогда
%   (n + 1)^2 - n^2 = 2n + 1 =
%   (x + 1)^2 + a(x + 1) + b - x^2 - ax - b = 2x + 1 + a
% значит n = x + a/2, в частности, a четно и равно 2c
% из (x + c)^2 = x^2 + 2cx + b имеем b = c^2, то есть у нас полный квадрат

\item
В~волейбольном турнире с~участием $73$~команд каждая команда сыграла с~каждой
по~одному разу.
В~конце турнира все команды разделили на~две группы так, что любая команда
первой группы одержала $n$~побед, а~любая команда второй группы~--- ровно
$m$~побед.
Могло~ли оказаться, что $m \neq n$?
% это задача 11.6 отсюда http://olympiads.mccme.ru/vmo/2012/iii-2.pdf

\item
Биссектрисы $AL$, $BM$ и~$CN$ треугольника $ABC$ пересекаются в~точке~$O$.
Какой из~отрезков $LO$, $MO$ и~$NO$ наибольший, если
$\angle A > \angle B > \angle C$?
% обозначим углы треугольника соответсвенно как 2\alpha, 2\beta, 2\gamma
% напишем, например, для BON и BOL теорему синусов
% OB / \sin(2\alpha + \gamma) = ON / \sin \beta
% OB / \sin(2\gamma + \alpha) = OL / \sin \beta
% значит ON / OL = \sin(2\gamma + \alpha) / \sin(2\alpha + \gamma), откуда все
% и ясно

\item
Пусть $p > 5$~--- простое число.
Из~десятичной записи числа $1 / p$ выкинули $2016$-ю цифру после запятой
и~получили число, представимое в~виде несократимой дроби $a / b$.
Докажите, что $b$ делится на~$p$.
% можно записать два уравнения
% 1/p = x / 10^k + y / 10^{k + 1} + t
% a/b = x / 10^k + 10t
% умножаем первое на 10 и вычитаем второе
% 10/p - a / b = (10b - ap) / bp = 9x + y/ 10^k
% так как p взаимно просто с 10, средняя дробь сократима на p, то есть 10b
% кратно p, а значит и b кратно p

\end{problems}


\subsection*{Вывод}

\begin{problems}

\item
Множество натуральных чисел разбито на~$2016$ бесконечных попарно
не~пересекающиеся арифметические прогрессий.
Верно~ли, что у~каждой из~этих прогрессий разность прогрессии не~меньше первого
члена прогрессии?
% Да, докажем это.
% Пусть есть прогрессия, для которой это не так, она имеет разность d
% и первый член a
% Пусть b - наибольшее число не в этой прогрессии сравнимое с a по модулю d
% пусть b лежит в прогрессии со разностью h
% тогда в этой же прогрессии лежит и число b + hd, которое сравнимо с a
% по модулю d и  превосходит b.
% Противоречие

\item
Даны положительные числа $b$ и $c$.
Докажите неравенство:
\[
    (b - c)^{2011} (b + c)^{2011} (c - b)^{2011}
\geq
    (b^{2011} - c^{2011}) (b^{2011} + c^{2011}) (c^{2011} - b^{2011})
\]
% решение: 11.8 отсюда
% http://olympiads.mccme.ru/vmo/2011/iii-2.pdf

\item
Точка~$O$ лежит внутри ромба $ABCD$.
Угол $\angle DAB$ равен $110^{\circ}$.
Углы $AOD$ и $BOC$ равны $80^{\circ}$ и $100^{\circ}$ соответственно.
Чему может быть равна величина угла $AOB$?
% решение: 9.5, 2008 год отсюда
% http://www.mccme.ru/free-books/olymp/mmo1993.pdf

%\item
%Назовем \emph{лестницей высоты~$n$} фигуру, состоящую из~всех клеток квадрата
%$n \times n$, лежащих не~выше диагонали.
%Сколькими различными способами можно разбить лестницу высоты~$n$ на~несколько
%прямоугольников, стороны которых идут по~линиям сетки, а~площади попарно
%различны?

\end{problems}

