% $date: 2016-01-12
% $timetable:
%   g9r2:
%     2016-01-12:
%       1:
%       2:
%   g9r1:
%     2016-01-12:
%       1:
%       2:

\section*{Заключительная олимпиада}

% $authors:
% - Алексей Александрович Пономарёв
% - Юлий Васильевич Тихонов

\subsection*{Довывод}

\begin{problems}

\item
Докажите, что среди любых $100$ трехзначных чисел найдутся четыре числа
с~одинаковой суммой цифр.

\item
Вася и~Даша едят шоколадку с~пятью продольными и~восемью поперечными
углублениями, по~которым ее можно ломать (всего $9 \times 6 = 54$ дольки).
Каждый из~них по~очереди (Вася начинает) отламывает от~оставшейся шоколадки
полоску ширины $1$ и~съедает.
Когда кто-то разламывает полоску ширины~$2$ на~две полоски ширины~$1$, она
съедает одну полоску, а~вторая полностью достается другому.
Докажите, что Вася может съесть хотя~бы на~$6$ долек больше, чем Даша.

\item
Вася нашел два числа $a$ и~$b$ таких, что $(a^3 - b^3) = 2$
и~$(a^5 - b^5) \geq 4$.
Докажите, что $a^2 + b^2 \geq 2$.

%\item
%Вася хочет найти четыре различных натуральных числа $x$, $y$, $z$, $t$,
%удовлетворяющих уравнению
%\[
%   x^x + y^y = z^z + t^t
%\, . \]
%Докажите, что ему не~удастся это сделать.

\item
Окружности $S_1$ и~$S_2$ с~центрам $O_1$ и~$O_2$ соответственно пересекаются
в~точках $A$ и~$B$.
Касательные к~$S_1$ и~$S_2$ в~точке~$A$ пересекают отрезки $B O_2$ и~$B O_1$
в~точках $K$ и~$L$ соответственно.
Докажите, что $KL \parallel O_1 O_2$.

\item
Вася выписал некоторый приведённый квадратный трехчлен $P(x)$.
Оказалось, что у~него есть общий корень с~многочленом $P(P(P(x)))$.
Докажите, что $P(0) \cdot P(1) = 0$.

\end{problems}


\subsection*{Вывод}

\begin{problems}

\item
Себастьян рассмотрел всевозможные наборы чисел $(x_{1}, x_{2}, \ldots, x_{20})$
такие, что каждое из~них лежит на~отрезке $[0;
1]$, и~равны произведения
\(
    x_{1} \cdot x_{2} \cdot \ldots \cdot x_{20}
=
    (1 - x_{1}) \cdot (1 - x_{2}) \cdot \ldots \cdot (1 - x_{20})
\).
Он выбрал среди таких наборов один, для которого значение этих произведений
максимально.
Найдите этот набор.

\item
Себастьян написал на~доске уравнение вида $x^2 + p x + q = 0$ с~целыми
ненулевыми коэффициентами $p$ и~$q$.
После этого к~доске иногда подходили другие школьники, стирали уравнение
и~выписывали уравнение такого~же вида, корнями которого были~бы коэффициенты
стертого уравнения.
В~какой-то момент на~доске оказалось написано исходное уравнение.
Что это могло быть за~уравнение?

\item
Четырехугольник~$ABCD$ вписан в~окружность, $I$~--- центр окружности, вписанной
в~треугольник $ABD$.
Найдите наименьшее значение $BD$, если $AI = BC = CD = 2$.

\end{problems}

