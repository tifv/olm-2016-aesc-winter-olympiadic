% $date: 2016-01-08
% $timetable:
%   g9r2:
%     2016-01-08:
%       1:
%   g9r1:
%     2016-01-08:
%       1:

\section*{Вступительная олимпиада}

% $authors:
% - Юлий Васильевич Тихонов

\begingroup
    \providecommand\ifsolutions{\iffalse}
    \long\def\solution#1{\ifsolutions\emph{Решение.} {#1}\fi}
    \def\abs#1{\lvert #1 \rvert}

\subsection*{Довывод}

\begin{problems}

\item
Какую минимальную сумму цифр может иметь натуральное число, делящееся на~$99$?
% http://problems.ru/view_problem_details_new.php?id=35790

  \solution{%
     Понятно, что сумма цифр должна делится на~$9$.
     Но~если она равна $9$, то~признак делимости на~$11$ не~может быть выполнен
     (если сумма цифр на~четных позициях $a$, а~на~нечетных $b$,
     то~$a + b = 9$, откуда $\abs{a - b} < 11$, то~есть $a - b = 0$ из~признака
     делимости на~$11$, но~тогда $a = b = 4{,}5$).
     Значит, минимальная сумма цифр~--- $18$.}

\item
Что больше: $2011^{2011} + 2009^{2009}$ или $2011^{2009} + 2009^{2011}$?

  \solution{%
    Вычтем из~первого второе, получим
    \(
        2011^{2009} \cdot (2011^2 - 1) - 2009^{2009} \cdot (2009^{2} - 1)
    \).
    Сомножители первого произведения соответственно больше сомножителей
    второго.}

\item
Одна из~сторон вписанного четырехугольника является диаметром окружности.
Докажите, что проекции сторон, прилегающих к~этой стороне, на~прямую, задающую
четвертую сторону, равны между собой.
% http://problems.ru/view_problem_details_new.php?id=53617

  \solution{%
    Обозначим четырехугольник $ABCD$, где $AB$~--- диаметр окружности;
    а~центр окружности пусть будет $O$.
    Проекции на~$CD$ обозначим соответственно $A'$, $B'$, $O'$.
    Тогда $O'$~--- середина $CD$ (центр окружности проецируется в~середину
    хорды), и~$O'$~--- середина $A'B'$ (середина отрезка проецируется
    в~середину).
    Отсюда легко следует, что $A'D = B'C$.}

\item
Двое по~очереди выписывают на~доску натуральные числа от~$1$ до~$1000$.
Первым ходом первый игрок выписывает на~доску число~$1$.
Затем очередным ходом на~доску можно выписать либо число~$2 a$, либо число
$a + 1$, если на~доске уже написано число~$a$.
При этом запрещается выписывать числа, которые уже написаны на~доске.
Выигрывает тот, кто выпишет на~доску число $1000$.
Кто выигрывает при правильной игре?
% http://problems.ru/view_problem_details_new.php?id=110141

  \solution{%
    Легко заметить, что игрок, выписавший любое из~чисел $500$ и~$999$,
    немедленно проиграет.
    Пусть $A$~--- множество всех чисел от~$1$ до~$1000$, кроме $1000$, $500$,
    $999$ и $501$.
    Тогда у~первого игрока появляется <<оттягивающая>> стратегия:
    если второй игрок выписал одно из~чисел $500$ или $999$, то~первый
    выписывает $1000$;
    в~противном случае он выписывает любое (например, наименьшее) еще
    не~выписанное число из~$A$.
    Он может это сделать, так как к~его ходу только четное количество чисел
    из~$A$ будет выписано, а~всего их там нечетно; число $501$ при этом
    не~будет выписано никогда.}

\item
Через центры некоторых клеток шахматной доски $8 \times 8$ проведена замкнутая
несамопересекающаяся ломаная.
Каждое звено ломаной соединяет центры соседних по~горизонтали, вертикали или
диагонали клеток.
Докажите, что в~ограниченном ею многоугольнике общая площадь черных частей
равна общей площади белых частей.
% http://www.problems.ru/view_problem_details_new.php?id=116542

  \solution{%
    Проведем всевозможные отрезки, которые могут быть звеньями нашей ломаной
    согласно условию.
    Заметим, что они образовали квадрат, разбитый на~маленькие треугольнички.
    Очевидно, что каждый такой треугольничек либо целиком окажется внутри
    ломаной, либо целиком снаружи;
    следовательно, наш многоугольник разобьется на~такие треугольнички.
    Осталось заметить, что в~каждом треугольничке белого и~черного поровну.}

\end{problems}


\subsection*{Вывод}

\begin{problems}

\item
Окружности с~центрами $O_1$ и~$O_2$ пересекаются в~точкaх $M$ и~$N$.
Прямые $O_1 M$ и~$O_2 M$ пересекают первую окружность в~точках $A_1$ и~$A_2$,
а~вторую в~точках $B_1$ и`$B_2$.
Докажите, что прямые $A_1 A_2$, $B_1 B_2$ и~$MN$ пересекаются в~одной точке.

  \solution{%
    Заметим, что указанные прямые являются высотами треугольника $A_1 B_2 M$.}

\item
Известно, что $0 < a, b, c, d < 1$
и~$a b c d = (1 - a) (1 - b) (1 - c) (1 - d)$.
Докажите, что $(a + b + c + d) - (a + c) (b + d) \geq 1$.

  \solution{%
    Искомое неравенство нетрудно переписать в~виде
    $(1 - a - c) (1 - b - d) \leq 0$.
    Таким образом, нужно доказать, что числа $(1 - a - c)$ и~$(1 - b - d)$
    находятся по~разные стороны от~$0$ (или одно из~них равно $0$).

    С~другой стороны, перепишем соотношение, данное по~условию, как
    \[
        \frac{(1 - a) (1 - c)}{a c}
        \cdot
        \frac{(1 - b) (1 - d)}{b d}
    =
        1
    \, . \]
    Раскрыв скобки в~числителях и~отделив $1$ от~каждой дроби, получаем
    \[
        \left( \frac{1 - a - c}{a c} + 1 \right)
        \cdot
        \left( \frac{1 - b - d}{b d} + 1 \right)
    =
        1
    \, . \]
    Очевидно, сомножители взаимно обратны и~находятся по~разные стороны от~$1$.
    Но~тогда $(1 - a - c)$ и~$(1 - b - d)$ находятся по~разные стороны от~$0$.}

\item
Ко~дню Российского Флага продавец украсил витрину $12$ горизонтальными
полосками ткани трех цветов.
При этом он выполняет два условия:
\\
\textsf{1.} одноцветные полосы не~должны висеть рядом;
\\
\textsf{2.} каждая синяя полоса должна висеть между белой и~красной.
\\
Сколькими способами он может это сделать?

  \solution{%
    Пусть $R_{n}$~--- количество способов расставить $n$ полосок по~указанным
    правилам, если первая полоска белая.
    Заметим, что $R_{1} = R_{2} = 1$.
    Также заметим, что $R_{n+2} = R_{n+1} + R_{n}$.
    Действительно, из~$n + 2$ полосок предпоследняя либо синяя, либо нет.
    В~первом случае первые $n$ полосок образуют одну из~$R_{n}$ расстановок,
    соответствующих правилам, а~последние две определяются однозначно.
    Во~втором случае первые $n + 1$ полос образуют одну из~$R_{n+1}$
    расстановок, а~оставшаяся определяется однозначно.
    Отсюда следует, что $R_{n}$~--- это $n$-е число Фибоначчи.
    Ответ в~два раза больше, так как расстановок, начинающихся с~красной
    полоски, еще столько~же.
    Это $288$.}

\end{problems}

\endgroup % \def\ifsolutions \def\solution \def\abs

