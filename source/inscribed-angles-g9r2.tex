% $date: 2016-01-11
% $timetable:
%   g9r2:
%     2016-01-11:
%       1:

% $caption:
%   Вписанные углы

\section*{Вписанные уголочки и дужки}

% $authors:
% - Алексей Александрович Пономарёв

\subsection*{Теоретическая часть}

\begin{problems}
    \makeatletter
        \renewcommand\problem@format[2]{\begingroup\sffamily
            #1.\rlap{\!#2}\endgroup}
        \renewcommand\subproblem@format[2]{\text{\begingroup\sffamily
            #1\rlap{#2}%
            \phantom{#2}%
        .%
        \endgroup}}
    \makeatother
    \small

\itemy{0}
\subproblem
Пусть точки $A$, $B$ и~$C$ расположены на~окружности с~центром~$O$.
Тогда угол $AOB$ называется \emph{центральным,} а~угол $ACB$~---
\emph{вписанным.}
Пусть угол $\angle AOB = 2 \alpha$.
Докажите, что $\angle ACB = \alpha$ если $C$ и~$O$ находятся по~одну сторону
от~прямой~$AB$, и~$\angle ACB = (180^{\circ} - \alpha)$ -- если по~разные.
\\
\subproblem
Отрезки $AC$ и~$BD$ пересекаются в~единственной точке, отличной
от~$A$, $B$, $C$, $D$.
Докажите, что точки $A$, $B$, $C$, $D$ лежат на~одной окружности тогда и~только
тогда, когда $\angle ABD = \angle ACD$.
\\
\subproblem
Докажите, что четырехугольник является вписанным в~том и~только в~том случае,
когда сумма его противоположных углов равна $180^{\circ}$ (четырехугольник
$ABCD$~--- вписанный, если точки $A$, $B$, $C$, $D$ лежат на~одной окружности).

\item
Высоты $A A_1$, $B B_1$ и~$C C_1$ треугольника $ABC$ пересекаются в~точке~$H$.
Докажите, что
\\
\subproblem четырехугольник $A B_1 H C_1$ вписанный;
\\
\subproblem $\angle BAH = \angle H B_1 A_1$;
\\
\subproblem $A A_1$~--- биссектриса треугольника $A_1 B_1 C_1$.

\item
Две окружности пересекаются в~точках $M$ и~$K$.
Через $M$ и~$K$ проведены прямые $AB$ и~$CD$ соответственно, пересекающие
первую окружность в~точках $A$ и~$C$, вторую в~точках $B$ и~$D$.
Докажите, что $AC \parallel BD$.

\item
\subproblem
Вершина~$A$ остроугольного треугольника $ABC$ соединена отрезком с~центром~$O$
описанной окружности.
Из~вершины~$A$ проведена высота~$A A_1$.
Докажите, что $\angle B A A_1 = \angle O A C$.
\\
\subproblem
Докажите, что \emph{ортоцентр} (точка пересечения высот) $H$
треугольника $ABC$, отраженный относительно стороны~$BC$, попадает в~точку
на~описанной окружности треугольника $ABC$.
\\
\subproblem
Докажите, что ортоцентр~$H$, отраженный относительно середины~$BC$, попадает
в~точку на~описанной окружности треугольника $ABC$, причем диаметрально
противоположную $A$.

\item
\subproblem
Прямая~$l$ проходит через точку~$A$, лежащую на~окружности~$\omega$
с~центром в~точке~$O$.
Докажите, что прямая~$l$ является касательной к~окружности~$\omega$ тогда
и~только тогда, когда отрезок~$OA$ перпендикулярен $l$ (прямая является
касательной к~окружности, если эта прямая и~эта окружность имеют ровно одну
общую точку).
\\
\subproblem
Прямая~$l$ касается окружности~$\omega$ в~точке~$A$.
Точки $C$, $D$ на~прямой~$l$ и~хорда~$AB$ таковы, что угол $\angle BAC$ острый,
а~угол $\angle BAD$ тупой.
Докажите, что угол $\angle BAC$ равен половине угловой величины
меньшей дуги~$AB$, а~угол $\angle BAD$~--- половине угловой величины
большей дуги~$AB$ (угловая величина дуги~$AB$ равна соответствующему
центральному углу $\angle AOB$, содержащему данную дугу).

\item
\subproblem
Из~точки~$A$, лежащей вне окружности, выходят лучи $AB$ и~$AC$, пересекающие
эту окружность.
Докажите, что величина угла $BAC$ равна полуразности угловых величин дуг
окружности, заключенных внутри этого угла.
\\
\subproblem
Хорды одной окружности $AB$ и~$CD$ пересекаются в~точке~$K$.
Докажите, что $\angle AKC$ равен полусумме угловых величин дуг $AC$ и~$BD$
(дуг, не~содержащих других отмеченных точек).

\end{problems}

\clearpage\resetproblem


\subsection*{Задачи для сдачи}

\begin{problems}

\item
Внутри треугольника $ABC$ взята точка~$P$ так, что
$\angle BPC = \angle A + 60^{\circ}$,
$\angle APC = \angle B + 60^{\circ}$,
$\angle APB = \angle C + 60^{\circ}$.
Прямые $AP$, $BP$ и~$CP$ пересекают описанную окружность треугольника $ABC$
в~точках $A_1$, $B_1$ и~$C_1$ соответственно.
Докажите, что треугольник $A_1 B_1 C_1$~--- правильный.

\item
Окружности $S_1$ и~$S_2$ пересекаются в~точке~$A$.
Через точку~$A$ проведена прямая, пересекающая $S_1$ в~точке~$B$, а~$S_2$
в~точке~$C$.
В~точках $C$ и~$B$ проведены касательные к~окружностям, пересекающиеся
в~точке~$D$.
Докажите, что угол $\angle BDC$ не~зависит от~выбора прямой, проходящей
через $A$.

\item
Внутри квадрата $ABCD$ отметили точку~$E$ так, что
$\angle ECD = \angle EAC = \alpha$.
Найдите $\angle ABE$.

\item
$ABCD$~--- вписанный четырехугольник.
Лучи $AB$, $DC$ пересекаются в~точке~$P$, лучи $BC$, $AD$~--- в~точке~$Q$.
Докажите, что биссектрисы углов $APD$ и~$AQB$ перпендикулярны.

\item
$ABCD$~--- вписанный четырехугольник, $K$~--- середина дуги~$AB$,
не~содержащей точек $C$ и~$D$.
$P$ и~$Q$~--- точки пересечения пар хорд $CK$ и~$AB$, $DK$ и~$AB$
соответственно.
Докажите, что четырехугольник $CPQD$~--- вписанный.

\item
Точки $A$, $B$, $M$, $N$ лежат на~окружности в~указанном порядке.
Пусть $A_1$, $B_1$~--- такие точки на~окружности, что
$N A \perp M B_1$, $N B \perp M A_1$.
Докажите, что $A A_1 \parallel B B_1$.

\item
$O$~--- центр описанной окружности равнобокой трапеции $ABCD$ с~боковой
стороной $AB$, а~$K$~--- точка пересечения ее диагоналей.
Докажите, что точки $A$, $B$, $K$, $O$ лежат на~одной окружности.

\item
Две окружности пересекаются в~точках $A$, $B$;
на~первой из~этих окружностей отмечена точка~$C$.
Прямые $CA$, $CB$ пересекают вторую окружность вторично в~точках $E$, $F$.
Докажите, что касательная к~первой окружности, восстановленная в~точке~$C$,
параллельна прямой~$EF$.

\item
Точки $A$, $B$, $C$, $D$ лежат на~одной окружности в~указанном порядке.
$K$, $L$, $M$, $N$~--- середины дуг $AB$, $BC$, $CD$, $DA$, не~содержащих
внутри четырех исходных точек.
Докажите, что $KM \perp LN$.

\item
В~треугольнике $ABC$ проведены медианы $B B_1$, $C C_1$.
Оказалось, что $\angle A B B_1 = \angle A C C_1$.
Докажите, что $AB = AC$.

\item
В~треугольнике $ABC$ проведена биссектриса~$AL$.
$K$~--- такая точка на~отрезке~$AC$, что $\angle BAC = \angle CLK$.
Докажите, что $BL = LK$.

\item\emph{Лемма о~трилистнике, она~же лемма о~трезубце.}
Пусть $I$~--- точка пересечения биссектрис треугольника $ABC$,
$A_0$~--- середина дуги~$BC$ описанной окружности треугольника $ABC$,
не~содержащей точку~$A$.
Докажите, что $A_0 B = A_0 I = A_0 C$.

\item
В~неравнобедренном треугольнике $ABC$ проведена биссектриса~$AD$.
$E$~--- точка пересечения прямой~$BC$ с~касательной к~описанной окружности
исходного треугольника, восстановленной в~точке~$A$.
Докажите, что $AE = DE$.

\end{problems}

