% $date: 2016-01-09
% $timetable:
%   g9r2:
%     2016-01-09:
%       1:

% $caption:
%   Комбинаторика. Оценка + пример

\section*{Оценка + пример}

% $authors:
% - Алексей Александрович Пономарёв

\emph{Решение большинства задач на~нахождение чего-нибудь наибольшего
и~наименьшего состоит из~двух частей: оценки и~примера.}

\begin{problems}

\itemy{0}
Какое наибольшее количество
\\
\subproblem ладей;
\qquad
\subproblem королей;
\qquad
\subproblem слонов;
\qquad
\subproblem коней
\\
можно расставить на~шахматной доске так, чтобы они не~били друг друга?

\end{problems}

\subsection*{Задачи}

\begin{problems}

\item
Том Сойер взялся покрасить очень длинный забор, соблюдая условия: любые две
доски, между которыми ровно две, ровно три или ровно пять досок, должны быть
окрашены в~разные цвета.
Какое наименьшее количество красок потребуется Тому для этой работы?

\item
Какое наименьшее количество трехклеточных уголков можно разместить
в~квадрате $8 \times 8$ так, чтобы в~этот квадрат больше нельзя было поместить
ни~одного такого уголка?

\item
Какое наибольшее количество клеток можно отметить на~шахматной доске так, чтобы
с~любой из~них на~любую другую отмеченную клетку можно было пройти ровно двумя
ходами шахматного коня?

\item
Какое наименьшее количество звеньев может быть в~замкнутой ломаной, каждое
звено которой пересекается ровно с~одним из~остальных звеньев?

\item
На~окружности расставлены $2013$ чисел, каждое из~которых равно $1$ или $-1$,
причем не~все числа одинаковы.
Рассмотрим всевозможные десятки подряд стоящих чисел.
Найдем произведение чисел в~каждом десятке и~сложим их.
Какая наибольшая сумма может получится?

\item
Числа от~$1$ до~$9$ расставили в~клетки таблицы $3 \times 3$ так, что во~всех
четырех квадратиках $2 \times 2$ сумма чисел одинаковы.
Какие значения может принимать эта сумма?

\item
Десять футбольных команд сыграли каждая с~каждой по~одному разу.
В~результате у~каждой команды оказалось ровно по~$x$ очков.
Каково наибольшее возможное значение $x$?
За~победу дается $3$~очка, за~ничью~--- $1$~очко, за~поражение~--- $0$~очков.

\item
Какое наименьшее количество множителей надо вычеркнуть из~произведения $10!$,
чтобы полученное произведение оканчивалось на~цифру~$2$?

\item
В~классе $25$~учеников.
Известно, что у~любых двух девочек класса количество друзей-мальчиков из~этого
класса не~совпадает.
Какое наибольшее количество девочек может быть в~этом классе?

\item
За~круглым столом сидят $2015$ человек, каждый из~них~--- либо рыцарь, либо
лжец.
Рыцари всегда говорят правду, лжецы всегда лгут.
Им раздали по~одной карточке, на~каждой карточке написано по~числу;
при этом все числа на~карточках различны.
Посмотрев на~карточки соседей, каждый из~сидящих за~столом сказал:
<<Мое число больше, чем у~каждого из~двух моих соседей>>.
После этого $k$ из~сидящих сказали:
<<Мое число меньше, чем у~каждого из~двух моих соседей>>.
При каком наибольшем $k$ это могло случиться?

\item
Назовем натуральное число интересным, если сумма его цифр~--- простое число.
Какое наибольшее количество интересных чисел может быть среди пяти подряд
идущих натуральных чисел?

\item
После просмотра фильма зрители по~очереди оценивали фильм целым числом баллов
от~$0$ до~$10$.
В~каждый момент времени рейтинг фильма вычислялся как сумма всех выставленных
оценок, деленная на~их количество.
В~некоторый момент времени $T$ рейтинг оказался целым числом, а~затем с~каждым
новым проголосовавшим зрителем он уменьшался на~единицу.
Какое наибольшее количество зрителей могло проголосовать после момента~$T$?

\item
Учитель записал Пете в~тетрадь четыре различных натуральных числа.
Для каждой пары этих чисел Петя нашел их наибольший общий делитель.
У~него получились шесть чисел: $1$, $2$, $3$, $4$, $5$ и~$N$, где $N > 5$.
Какое наименьшее значение может иметь число~$N$?

\item
Имеются $2013$ карточек, на~которых написана цифра~$1$, и~$2013$ карточек,
на~которых написана цифра~$2$.
Вася складывает из~этих карточек $4026$-значное число.
За~один ход Петя может поменять местами некоторые две карточки и~заплатить Васе
$1$~рубль.
Процесс заканчивается, когда у~Пети получается число, делящееся на~$11$.
Какую наибольшую сумму может заработать Вася, если Петя стремится заплатить как
можно меньше?

\item
В~клетках доски $8 \times 8$ расставлены числа $1$ и~$-1$
(в~каждой клетке~--- по~одному числу).
Рассмотрим всевозможные расположения фигурки\enspace
\jeolmfigure[height=1.5ex]{tetramino-T}%
\enspace
на~доске
(фигурку можно поворачивать, но~ее клетки не~должны выходить за~пределы доски).
Назовем такое расположение неудачным, если сумма чисел, стоящих в~четырех
клетках фигурки, не~равна~$0$.
Найдите наименьшее возможное число неудачных расположений.

\item
На~доске написаны несколько чисел.
Известно, что квадрат любого записанного числа больше произведения любых двух
других записанных чисел.
Какое наибольшее количество чисел может быть на~доске?

\item
Прямую палку длиной $2$~метра распилили на~$N$~палочек, длина каждой из~которых
выражается целым числом сантиметров.
При каком наименьшем $N$ можно гарантировать, что, использовав все получившиеся
палочки, можно, не~ломая их, сложить контур некоторого прямоугольника?

\end{problems}

