% $date: 2016-01-10
% $timetable:
%   g9r2:
%     2016-01-10:
%       3:

\section*{Квадратный трёхчлен --- 3}

% $authors:
% - Ольга Дмитриевна Телешева

\begingroup
    \def\abs#1{\lvert #1 \rvert}

\begin{problems}

\item
Пусть у~трехчлена $ax^2 + bx + c$ с~целыми $a,$ $b,$ $c$ есть два корня, один
из~которых рациональный.
\\
\subproblem
Верно~ли, что второй корень тоже рациональный?
\\
\subproblem
Пусть первый корень целый.
Верно~ли, что второй корень целый?
\\
\subproblem
Пусть первый корень целый и~$a = -1$.
Верно~ли, что второй корень целый?
\\
\subproblem
Пусть $a = 1$, а~про первый корень известна только рациональность.
Докажите, что всё равно оба корня целые

\item
Может~ли у~квадратного трехчлена с~целыми коэффициентами дискриминант равняться
2015?

\item
Квадратный трехчлен $f(x) = ax^2 + bx + c$ принимает в~точках $1/a$ и~$c$
значения разных знаков.
Докажите, что корни трехчлена $f(x)$ имеют разные знаки.

\item
Даны два квадратных трехчлена, имеющих корни.
Известно, что если в~них поменять местами коэффициенты при $x^2$, то~получатся
трехчлены, не~имеющие корней.
Докажите, что если в~исходных трехчленах поменять местами коэффициенты при $x$,
то~получатся трехчлены, имеющие корни.

\item
Рассмотрим графики функций $y = x^2 + px + q$, которые пересекают оси
координат в~трех различных точках.
Докажите, что все окружности, описанные около треугольников с~вершинами в~этих
точках, имеют общую точку.

\item
Есть две параболы $y = x^2 + x - 41$ и~$x = y^2 + y - 40$.
Докажите,что точки их пересечения лежат на~одной окружности.

\item
Ненулевые числа $a$ и~$b$ таковы, что уравнение $a (x - a)^2 + b (x - b)^2 = 0$
имеет единственное решение.
Докажите, что $\abs{a} = \abs{b}$.

\end{problems}

\endgroup % \def\abs

