% $date: 2016-01-10
% $timetable:
%   g9r2:
%     2016-01-10:
%       2:

\section*{Квадратный трёхчлен --- 2. Теорема Виета}

% $authors:
% - Ольга Дмитриевна Телешева

\begin{problems}

\item
При каких $p$ и~$q$ уравнению $x^2 + px + q = 0$ удовлетворяют два различных
числа $2p$ и~$p + q?$

\item
При каких значениях параметра $a$ сумма квадратов корней уравнения
$4 x^2 - 28 x + a = 0$ равна $22{,}5$?

\item
Пусть $x_1$ и~$x_2$ -- корни квадратного уравнения $x^2 - 3 x - 5 = 0$.
Составьте квадратное уравнение, корнями которого являются числа:
\\
\subproblem
$x_1 + 1 / x_1$ и~$x_2 + 1 / x_2$;
\qquad
\subproblem
$x_1 + 1 / x_2$ и~$x_2 + 1 / x_1$.

\item
Вася отвечает теорему Виета:
<<Сумма трех коэффициентов квадратного трехчлена
равна одному из~его корней, а~произведение -- другому>>.
\\
Экзаменатор: <<Неверно>>.
\\
Вася:
<<Как~же неверно?
Я проверил для случайно выбранного трехчлена, и~всё получилось>>.
\\
Какой это мог быть трехчлен, если его коэффициенты~--- целые числа?

\item
При каком значении параметра~$m$ сумма квадратов корней уравнения\enspace
$x^2 - (m + 1) x + m - 1 = 0$%
\enspace
является наименьшей?

\item
Известно, что корни уравнения $x^2 + px + q = 0$~--- целые числа,
а~$p$ и~$q$~--- простые числа.
Найдите $p$ и~$q$.

\item
На~рисунке изображен график функции $y = x^2 + ax + b$.
Известно, что прямая~$AB$ перпендикулярна прямой $y = x$.
Найдите длину отрезка $OC$.
\begin{center}
    \jeolmfigure[width=0.35\linewidth]{abc}
\end{center}

\item
Миша решил уравнение $x^2 + a x + b = 0$ и~сообщил Диме набор из~четырех
чисел~--- два корня и~два коэффициента этого уравнения
(но~не~сказал, какие именно из~них корни, а~какие~--- коэффициенты).
Сможет~ли Дима узнать, какое уравнение решал Миша, если все числа набора
оказались различными?

\end{problems}

