% $date: 2016-01-09
% $timetable:
%   g9r2:
%     2016-01-09:
%       2:

\section*{Квадратный трёхчлен --- 1}

% $authors:
% - Ольга Дмитриевна Телешева

\begin{problems}

\item
Верно~ли, что если $b > a + c > 0$, то~квадратное уравнение
$a x^2 + bx + c = 0$ имеет два корня?

\item
Стёпа нарисовал на~доске три параболы (см. рис.).
Катя утверждает, что уравнения этих парабол
$y = a x^2 + b x + c$, $y = bx^2 + c x + a$, $y = cx^2 + ax + b$ в~каком-то
порядке.
Может~ли это быть правдой при некоторых $a$, $b$ и~$c$?
\begin{center}
    \jeolmfigure[width=0.35\linewidth]{abc}
\end{center}

\item
Известно, что $c (a + b + c) < 0$.
Докажите, что уравнение $a x^2 + b x + c = 0$ имеет корни.

\item
Квадратный трехчлен $y = a x^2 + bx + c$ не~имеет корней и~$a + b + c > 0$.
Найдите знак коэффициента~$c$.

\item
При каких значениях параметра~$a$ один из~корней уравнения
$(a^2 + a + 1) x^2 + (2 a - 3) x + (a - 5) = 0$
больше $1$, а~другой~--- меньше $1$?

\item
Даны уравнения
\[
\text{(1)}\quad
    a x^2 + b x + c = 0
\qquad\text{и}\qquad
\text{(2)}\quad
    -a x^2 + b x + c = 0
\, . \]
Докажите, что если $x_1$ и~$x_2$~--- соответственно какие-либо корни уравнений
\text{(1)} и~\text{(2)}, то~найдется такой корень~$x_3$ уравнения
$0{,}5 a x^2 + b x + c$, что либо
$x_1 \leq x_3 \leq x_2$, либо $x_1 \geq x_3 \geq x_2$.

\item
\subproblem
Сколько общих точек могут иметь две параболы, являющиеся графиками квадратных
трехчленов в~одной и~той~же системе координат?
\\
\subproblem
Через точку~$A$ параболы проведены всевозможные прямые.
Сколько из~них имеют с~параболой только одну общую точку?
(Рассмотрите различные положения точки~$A$.)

\end{problems}


\subsection*{Домашнее задание}

\begin{problems}

\item
Про коэффициенты $a$, $b$, $c$ и~$d$ двух квадратных трехчленов
$x^2 + b x + c$ и~$x^2 + a x + d$ известно, что $0 < a < b < c < d$.
Могут~ли эти трехчлены иметь общий корень?

\item
Рассмотрим квадратичные функции $y = x^2 + p x + q$, для которых
$p - q = 2016$.
Покажите, что все параболы, являющиеся графиками этих функций, пересекаются
в~одной точке.

\end{problems}

