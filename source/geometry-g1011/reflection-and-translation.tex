% $date: 2016-01-11
% $timetable:
%   g1011:
%     2016-01-11:
%       2:

\section*{Симметрия + перенос}

% $authors:
% - Константин Георгиевич Евдокимов

\begin{problems}

\item
Отрезки $AB, BC$ и~$CD$ являются хордами одной окружности.
Точки $M,K$ и~$P$ – их середины соответственно.
Известно, что $\angle CKP = \alpha, \angle PCK = \beta$.
Найдите $\angle KMB$.

\item
Дан параллелограмм $ABCD$ и~точка $M$ внутри него.
Через точки $A$, $B$, $C$ и~$D$ проведены прямые параллельные $MC$, $MD$, $MA$
и~$MB$ соответственно.
Докажите, что полученные прямые пересекаются в~одной точке.

\item
Пусть $H$~--- ортоцентр треугольника $ABC$, докажите что радиусы описанных
окружностей треугольников $ABC$, $ABH$, $BCH$, $CAH$ равны.

\item
Дан параллелограмм $ABCD$.
Точка~$E$ вне параллелограмма такова, что $\angle CED = \angle AEB$.
Докажите, что $\angle EAD = \angle ECD$.

\item
Внутри параллелограмма $ABCD$ выбрана точка~$P$ так, что
$\angle APB + \angle CPD = 180^{\circ}$.
Докажите, что $\angle PBC = \angle PDC$.

\item
Дана окружность с~центром в~$O$.
Дана прямая~$l$, проходящая через $O$ и~точка~$C$ на~прямой~$l$ внутри
окружности.
Точки~$A$ и~$A_1$ взяты на~окружности в~одной полуплоскости относительно $OC$
так, что $AC$ и~$A_1 C$ образуют одинаковый угол с~прямой~$l$.
Отрезок~$A A_1$ продлили до~пересечения с~прямой~$l$~--- получили точку~$B$.
Докажите, что положение точки~$B$ не~зависит от~положения точки~$A$.

\item
Дан равнобедренный треугольник $ABC$ с~$\angle C = 120^{\circ}$.
Из~вершины~$C$ выпущены
два луча в~сторону~$AB$~--- угол между лучами $60^{\circ}$~---,
которые, отразившись от~нее в~точках $M$ и~$N$, пересекают стороны $AC$ и~$BC$
соответственно в~точках $A_1$ и~$B_1$.
Докажите, что площадь $CMN$ совпадает с~суммой площадей $A A_1 M$ и~$B B_1 N$.

\item
Треугольники $A B C$ и~$A A_1 B_1$ подобны.
Точка~$A_1$ лежит на продолжении $BC$ за точку~$C$.
Дано $\angle B = \angle B_1 = 90^{\circ}$;
треугольники не~накладываются друг на~друга.
Докажите, что центр описанной окружности треугольника $A A_1 C$ лежит
на~прямой~$A_1 B_1$.

\end{problems}

