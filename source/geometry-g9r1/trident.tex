% $date: 2016-01-09
% $timetable:
%   g9r1:
%     2016-01-09:
%       3:

% $caption:
%   Геометрия. Лемма о трезубце

\section*{Лемма о куриной лапке и критерий касания прямой и окружности}

% $authors:
% - Алексей Александрович Пономарёв

\begin{itemize}

\item\emph{Воспоминания из~начальной школы.}
Дана окружность.
Через точку~$N$ внутри этой окружности проведены две прямые.
Докажите, что угол между ними равен полусумме дуг, высекаемых прямыми
из~окружности.
А~что можно сказать, если точка $N$ вне окружности?
А~про угол между касательной и~хордой?

\item\emph{Критерий касания.}
Дан треугольник $ABC$ и~прямая~$CD$
(точки $D$ и~$B$ лежат в~разных полуплоскостях относительно прямой~$AC$).
Докажите, что окружность, описанная около треугольника $ABC$, касается
прямой~$CD$, если и~только если $\angle ABC = \angle ACD$.

\item\emph{Лемма о~трезубце.}
Середина дуги~$BC$ (не~содержащая точки~$A$) описанной окружности треугольника
$ABC$ равноудалена от~$B$, $C$, центра вписанной и~одного из~центров
невписанной окружности.

\end{itemize}


\subsection*{Задачи}

\begin{problems}

%тренировка к~критерию касания и~лемме о~трезубце
\item
В~треугольнике $ABC$ ($AB < BC$) точка~$I$~--- центр вписанной окружности,
$M$~--- середина стороны~$AC$, $N$~--- середина дуги $ABC$ описанной
окружности.
Докажите, что $\angle IMA = \angle INB$.
%\emph{(IV этап Всероссийской, 9 класс, 2005)}

%тренировка к~критерию касания прямой и~окружности
\item
Биссектрисы углов $A$ и~$C$ треугольника $ABC$ пересекают описанную окружность
этого треугольника в~точках $A_0$ и~$C_0$ соответственно.
Прямая, проходящая через центр вписанной окружности треугольника $ABC$
параллельно стороне~$AC$, пересекается с~прямой~$A_0 C_0$ в~точке~$P$.
Докажите, что прямая~$PB$ касается описанной окружности треугольника $ABC$.
%\emph{(IV этап Всероссийской, 9 класс, 2006)}

%тренировка к~лемме о~трезубце
\item
Треугольник $ABC$ вписан в~окружность~$S$.
Пусть $A_0$~--- середина дуги~$BC$ окружности~$S$, не~содержащей $A$;
$C_0$~--- середина дуги~$AB$, не~содержащей $C$.
Окружность~$S_1$ с~центром~$A_0$ касается $BC$,
окружность~$S_2$ с~центром~$C_0$ касается $AB$.
Докажите, что центр~$I$ вписанной в~треугольник $ABC$ окружности лежит на~одной
из~общих внешних касательных к~окружностям $S_1$ и~$S_2$.

%тренировка к~критерию касания прямой и~окружности
\item
Окружность, вписанная в~угол с~вершиной~$O$, касается его сторон в~точках
$A$ и~$B$,
$K$~--- произвольная точка на~меньшей из~двух дуг~$AB$ этой окружности.
На~прямой~$OB$ взята точка~$L$ такая, что прямые $OA$ и~$KL$ параллельны.
Пусть $M$~--- точка пересечения окружности~$\omega$, описанной около
треугольника $KLB$, с~прямой~$AK$, отличная от~$K$.
Докажите, что прямая~$OM$ касается окружности~$\omega$.

%тренировка к~критерию касания прямой и~окружности (кривая, ибо прямые углы)
\item
В~окружность вписан прямоугольный треугольник $ABC$ с~гипотенузой~$AB$.
Пусть $K$~--- середина дуги~$BC$, не~содержащей точки~$A$,
$N$~--- середина отрезка~$AC$,
$M$~--- точка пересечения луча~$KN$ с~окружностью.
В~точках $A$ и~$C$ проведены касательные к~окружности, которые пересекаются
в~точке~$E$.
Докажите, что угол $EMK$~--- прямой.

%тренировка к~лемме о~трезубце
\item
Точка~$I$~--- центр вписанной окружности треугольника $ABC$.
Внутри треугольника выбрана точка~$P$ такая,
что
\[
    \angle PBA + \angle PCA
=
    \angle PBC + \angle PCB
\]
Докажите, что $AP \geq AI$, причем равенство выполняется тогда и~только тогда,
когда $P$ совпадает с~$I$.

\item
Серединный перпендикуляр к~стороне~$AC$ остроугольного треугольника $ABC$
пересекает прямые $AB$ и~$BC$ в~точках $B_1$ и~$B_2$ соответственно,
а~серединный перпендикуляр к~стороне~$AB$ пересекает прямые $AC$ и~$BC$
в~точках $C_1$ и~$C_2$ соответственно.
Окружности, описанные около треугольников $B B_1 B_2$ и~$C C_1 C_2$
пересекаются в~точках $P$ и~$Q$.
Докажите, что центр окружности, описанной около треугольника $ABC$, лежит
на~прямой~$PQ$.

\end{problems}

