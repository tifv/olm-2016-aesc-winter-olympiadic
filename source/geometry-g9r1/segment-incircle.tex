% $date: 2016-01-10
% $timetable:
%   g9r1:
%     2016-01-10:
%       3:

\section*{Лемма Архимеда}

% $authors:
% - Алексей Александрович Пономарёв

\begin{itemize}

\item\emph{Лемма Архимеда.}
В~окружности проведена хорда~$AB$.
Другая окружность касается отрезка~$AB$ в~точке~$K$ и~окружности в~точке~$L$.
Докажите, что прямая~$KL$ проходит через середину дуги, дополняющей дугу $ALB$
до~окружности.

\end{itemize}

\subsection*{Задачи}

\begin{problems}

\item
В~окружности проведена хорда.
В~оба образовавшихся сегмента вписаны окружности.
Прямые, проходящие через точки касания этих окружностей с~данной и~хордой
повторно пересекают данную окружность в~точках $A$ и~$B$.
Докажите, что $AB$~--- диаметр данной окружности.

\item
На~дугах $AB$ и~$BC$ окружности, описанной около треугольника $ABC$, выбраны
соответственно точки $K$ и~$L$ так, что прямые $KL$ и~$AC$ параллельны.
Докажите, что центры вписанных окружностей треугольников $ABK$ и~$CBL$
равноудалены от~середины дуги $ABC$.

\item
На~диаметре~$AB$ окружности~$S$ взята точка~$K$ и~из~нее восстановлен
перпендикуляр, пересекающий $S$ в~точке~$L$.
Окружности $S_A$ и~$S_B$ касаются окружности~$S$, отрезка~$LK$ и~диаметра~$AB$,
а~именно, $S_A$ касается отрезка $AK$ в~точке $A_1$, $S_B$ касается отрезка
$BK$ в~точке $B_1$.
Докажите, что $\angle A_1 L B_1 = 45^{\circ}$.

\item
Внутри треугольника $ABC$ взята точка~$X$.
Прямая~$AX$ пересекает описанную окружность в~точке~$A_1$.
В~сегмент, отсекаемый стороной~$BC$, вписана окружность, касающаяся дуги~$BC$
в~точке~$A_1$, а~стороны $BC$~--- в~точке~$A_2$.
Точки $B_2$ и~$C_2$ определяются аналогично.
Докажите, что прямые $A A_2$, $B B_2$ и~$C C_2$ пересекаются в~одной точке.

\end{problems}

