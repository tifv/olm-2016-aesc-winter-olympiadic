% $date: 2016-01-11
% $timetable:
%   g9r1:
%     2016-01-11:
%       3:

% $caption:
%   Геометрия. Теорема Тебо

\section*{Лемма о сегменте и теорема Тебо\footnote{Sawayama-Thebault theorem.}}

% $authors:
% - Алексей Александрович Пономарёв

\begin{problems}

\item
\emph{Подсказки.}
\\
\subproblem
Пусть $AC$~--- хорда окружности, $B_1$~--- середина дуги~$AC$.
Прямая, проходящая через $B_1$, пересекает $AC$ в~точке $K$ и~окружность
в~точке $N$.
Докажите, что $B_1 K \cdot B_1 N = {B_1 C}^2$.
\\
\subproblem
Пусть $I$~--- центр окружности, вписанной в~треугольник $ABC$.
$B_1$~--- середина дуги~$AC$ окружности, описанной около треугольника $ABC$.
Прямая, проходящая через $B_1$, пересекает $AC$ в~точке~$K$ и~описанную
окружность в~точке~$N$.
Докажите, что $\angle BIN = \angle IKN$.

\item\emph{Лемма о~сегменте.}
Пусть $D$~--- точка на~стороне~$AC$ треугольника $ABC$.
Рассмотрим окружность, касающуюся отрезков $BD$, $DC$ и~окружности, описанной
около треугольника $ABC$.
Пусть $M$ и~$K$~--- точки касания этой окружности с~$BD$ и~$DC$ соответственно.
Докажите, что прямая~$MK$ проходит через точку~$I$~--- центр окружности,
вписанной в~треугольник $ABC$.

\item\emph{Теорема Виктора Тебо.}
Пусть $ABC$~--- произвольный треугольник.
$D$~--- произвольная точка на~стороне~$AC$.
$I_1$~--- центр окружности, касающейся отрезков $AD$, $BD$ и~описанной около
треугольника $ABC$ окружности.
$I_2$~--- центр окружности, касающейся отрезков $CD$, $BD$ и~описанной около
треугольника $ABC$ окружности.
Тогда отрезок~$I_1 I_2$ проходит через точку~$I$~--- центр окружности,
вписанной в~треугольник $ABC$, и~при этом
$I_1 I : I I_2 = \tg^2(\phi / 2)$, где $\phi = \angle BDA$.

\end{problems}

\subsection*{Задачи}

\begin{problems}

\item
Окружность касается продолжений сторон $CA$ и~$CB$ треугольника $ABC$, а~также
касается стороны~$AB$ этого треугольника в~точке~$P$.
Докажите, что радиус окружности, касающейся отрезков $AP$, $CP$ и~описанной
около этого треугольника окружности равен радиусу вписанной в~этот треугольник
окружности.

\item
Пусть треугольник $ABC$ вписан в~окружность~$\omega$.
$A_0$, $B_0$~--- точки на~сторонах $BC$ и~$CA$ соответственно, такие что прямая
$A_0 B_0$ параллельна $AB$.
В~сегменты, стягиваемые хордами $BC$ и~$CA$ окружности~$\omega$, не~содержащие
$A$ и~$B$ соответственно, вписаны окружности $\omega_A$, $\omega_B$, касающиеся
хорд $BC$ и~$CA$ в~точках $A_0$, $B_0$.
Докажите, что общая касательная к~окружностям $\omega_A$, $\omega_B$,
<<ближайшая>> к~$AB$, параллельна $AB$.

\end{problems}

