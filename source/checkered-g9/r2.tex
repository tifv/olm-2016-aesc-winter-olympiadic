% $date: 2016-01-08
% $timetable:
%   g9r2:
%     2016-01-08:
%       3:

\section*{Задачи на клетчатой бумаге}

% $authors:
% - Аскар Флоридович Назмутдинов

\begin{problems}

\item
Существует~ли треугольник с~вершинами в~узлах клетчатой бумаги, каждая сторона
которого длиннее $100$ клеточек, а~площадь меньше площади одной клеточки?

\item
В~прямоугольнике с~целыми сторонами $m$ и~$n$, нарисованном на~клетчатой
бумаге, проведена диагональ.
Через какое число узлов она проходит?
На~сколько частей эта диагональ делится линиями сетки?

\item
Дана бесконечная клетчатая бумага и~фигура, площадь которой меньше площади
клетки.
Докажите, что эту фигуру можно положить на~бумагу, не~накрыв ни~одной вершины
клетки.

\item
На~бесконечном листе клетчатой бумаги (размер клетки $1 \times 1$) укладываются
кости домино размером $1 \times 2$ так, что они накрывают все клетки.
Можно~ли при этом добиться того, чтобы любая прямая, идущая по~линиям сетки,
разрезала лишь конечное число костей?

\item
Можно~ли разбить клетчатую доску $12 \times 12$ на~уголки из~трех соседних
клеток так, чтобы каждый горизонтальный и~каждый вертикальный ряд клеток
пересекал одно и~то~же количество уголков?
(Ряд пересекает уголок, если содержит хотя~бы одну его клетку).

\item
В~парке растет $10\,000$ деревьев, посаженных квадратно-гнездовым способом
($100$ рядов по~$100$ деревьев).
Какое наибольшее число деревьев можно срубить, чтобы выполнялось следующее
условие: если встать на~любой пень, то~не~будет видно ни~одного другого пня?
(Деревья можно считать достаточно тонкими.)

\item
Узлы бесконечной клетчатой бумаги раскрашены в~три цвета.
Докажите, что существует равнобедренный прямоугольный треугольник с~вершинами
одного цвета.

\item
Лёша задумал двузначное число (от~$10$ до~$99$).
Гриша пытается его отгадать, называя двузначные числа.
Считается, что он отгадал, если одну цифру он назвал правильно, а~в~другой
ошибся не~более чем на~единицу (например, если задумано число $65$, то~$65$,
$64$ и~$75$ подходят, а~$63$, $76$ и~$56$~--- нет).
Может~ли Гриша гарантированно угадать число Лёши
\\
\subproblem за~$22$ попытки?
\qquad
\subproblem А за~$18$?

\item
Прямоугольник размером $1 \times k$ при всяком натуральном~$k$ будем называть
\emph{полоской.}
При каких натуральных $n$ прямоугольник размером $1995 \times n$ можно
разрезать на~попарно различные полоски?

\end{problems}

