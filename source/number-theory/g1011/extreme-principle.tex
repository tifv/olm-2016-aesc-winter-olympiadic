% $date: 2016-01-09
% $timetable:
%   g1011:
%     2016-01-09:
%       1:

\section*{Теория чисел. Принцип крайнего}

% $authors:
% - Глеб Александрович Погудин
% - Юлий Васильевич Тихонов
% - Наталия Евгеньевна Шавгулидзе

\subsubsection*{Крайнее число}

\begin{problems}

\item
На~доску выписаны $2011$ чисел.
Оказалось, что сумма любых трех выписанных чисел также является выписанным
числом.
Какое наименьшее количество нулей может быть среди этих чисел?
\emph{(2011, 10.6)}

\item
В~десятичной записи некоторого числа цифры расположены слева направо в~порядке
строгого убывания.
Может~ли это число быть кратным числу $111$?
% http://problems.ru/view_problem_details_new.php?id=116878

\item
Докажите, что числа от~$1$ до~$16$ можно записать в~строку, но~нельзя записать
по~кругу так, чтобы сумма любых двух соседних чисел была квадратом натурального
числа.
% http://problems.ru/view_problem_details_new.php?id=109927

%\item
%На~доске написано несколько натуральных чисел.
%Сумма любых двух из~них~--- натуральная степень двойки.
%Какое наибольшее число различных может быть среди чисел на~доске?
%% http://problems.ru/view_problem_details_new.php?id=116683

% на олимпиадку
%\item
%Докажите, что не~существует попарно различных натуральных чисел
%$x$, $y$, $z$, $t$, для которых было~бы справедливо соотношение
%\[
%    x^x + y^y = z^z + t^t
%\, . \]
%% http://problems.ru/view_problem_details_new.php?id=78507

\item
Решите в~натуральных числах уравнение $n^{x} + n^{y} = n^{z}$.
% http://problems.ru/view_problem_details_new.php?id=73766

\item
Найдите все тройки простых чисел $p$, $q$, $r$ такие, что четвертая степень
любого из~них, уменьшенная на~$1$, делится на~произведение двух остальных.
\emph{(2011, 9.7)}
% наименьшее из p, q, r

\item
Докажите, что в~бесконечной последовательности попарно различных натуральных
чисел, больших единицы, найдется бесконечное количество чисел, которые больше
своего номера в~этой последовательности.
% http://problems.ru/view_problem_details_new.php?id=97783

\end{problems}


\subsubsection*{Крайний делитель (или степень делителя)}

\begin{problems}

\item
Можно~ли при каком-то натуральном $k$ разбить все натуральные числа
от~$1$ до~$k$ на~две группы и~выписать числа в~каждой группе подряд в~некотором
порядке так, чтобы получились два одинаковых числа?
\emph{(2010, 9.3, 10.2)}
% наибольшая степень 10

\item
Дано $n$ попарно взаимно простых чисел, больших $1$ и~меньших $(2 n - 1)^{2}$.
Докажите, что среди них обязательно есть простое число.
% http://problems.ru/view_problem_details_new.php?id=34920

\item
Докажите, что каждое натуральное число является разностью двух натуральных
чисел, имеющих одинаковое количество простых делителей.
(Каждый простой делитель учитывается 1~раз, например, число~$12$ имеет два
простых делителя: $2$ и~$3$.)
% http://problems.ru/view_problem_details_new.php?id=110014

%\item
%По~кругу стоят $10^{1000}$ натуральных чисел.
%Между каждыми двумя соседними числами записали их наименьшее общее кратное.
%Могут~ли эти наименьшие общие кратные образовать $10^{1000}$ последовательных
%чисел (расположенных в~каком-то порядке)?
%% регион 2014 10.7
%% наибольшая степень 2

\item
\subproblem
Докажите, что числа $H_{n}$ не~являются целыми при $n > 1$:
\[
    H_{n}
=
    1 + \frac{1}{2} + \frac{1}{3} + \ldots + \frac{1}{n}
\;
.
\]
\subproblem
Докажите, что сумма всех чисел вида $1 / (m n)$, где $m$ и~$n$~--- натуральные
числа, $1 < m < n < 1986$, не~является целым числом.
% http://problems.ru/view_problem_details_new.php?id=34902

\end{problems}


\subsubsection*{Разнобой}

\begin{problems}

% это уже жесть, скорее всего
\item
Докажите, что из~любых шести четырехзначных чисел, взаимно простых
в~совокупности, всегда можно выбрать пять чисел, также взаимно простых
в~совокупности.
% http://problems.ru/view_problem_details_new.php?id=110137

\iffalse % BEGIN ЖЕСТЬ ЖЕСТЬ ЖЕСТЬ

% это жесть
\item
На~доске написаны $N \geq 9$ различных неотрицательных чисел, меньших единицы.
Оказалось, что для любых восьми различных чисел с~доски на~ней найдется такое
девятое, отличное от~них, что сумма этих девяти чисел целая.
При каких $N$ это возможно?
% http://problems.ru/view_problem_details_new.php?id=65241
% всерос 2015 9.8

% тоже жесть
\item
На~бесконечной в~обе стороны ленте бумаги выписаны все целые числа, каждое~---
ровно по~одному разу.
Могло~ли оказаться, что между каждыми двумя числами не~стоит их среднее
арифметическое?
(<<Между>> не~подразумевает <<подряд>>.)
% http://problems.ru/view_problem_details_new.php?id=111808

% решить и умереть
\item
Из~двухсот чисел: $1, 2, 3, \ldots, 199, 200$ выбрали одно число, меньшее $16$,
и~еще $99$~чисел.
Докажите, что среди выбранных чисел найдутся хотя~бы 2 таких, что одно из~них
делится на~другое.
% http://problems.ru/view_problem_details_new.php?id=76552

\fi % END ЖЕСТЬ ЖЕСТЬ ЖЕСТЬ

\end{problems}


