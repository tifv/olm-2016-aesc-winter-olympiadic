% $date: 2016-01-09
% $timetable:
%   g1011:
%     2016-01-09:
%       2:

\section*{Теория чисел. Принцип Дирихле}

% $authors:
% - Глеб Александрович Погудин
% - Юлий Васильевич Тихонов
% - Наталия Евгеньевна Шавгулидзе

\begingroup
    \def\abs#1{\lvert #1 \rvert}

\subsubsection*{Остатки}

\begin{problems}

\item
Докажите, что из~любых двенадцати натуральных чисел можно выбрать два,
разность которых делится на~$11$.

\item
\subproblem
В~ряд выписано пять натуральных чисел.
Докажите, что либо одно из~них делится на~$5$, либо сумма нескольких рядом
стоящих делится на~$5$.
\\
\subproblem
В~строку выписано $23$ натуральных числа (не~обязательно различных).
Докажите, что между ними можно так расставить скобки, знаки сложения
и~умножения, что значение полученного выражения будет делиться на~$2000$
нацело.

\item
Дана бесконечная последовательность цифр.
Докажите, что для любого натурального числа~$n$, взаимно простого
с~числом~$10$, можно указать такую группу стоящих подряд цифр
последовательности, что записываемое этими цифрами число делится на~$n$.

\item
Докажите, что внутри выпуклого пятиугольника с~вершинами в~узлах сетки есть
хотя~бы одна целая точка.

\end{problems}

\subsubsection*{Сдвиги}

\begin{problems}

\item
Имеется $2 k + 1$ карточек, занумерованных числами от~$1$ до~$2 k + 1$.
Какое наибольшее число карточек можно выбрать так, чтобы ни~один из~извлеченных
номеров не~был равен сумме двух других извлеченных номеров?
% alternative:
%\item
%Выбраны $1002$ различных натуральных числа, не~превосходящих $2000$.
%Доказать, что среди них можно найти три числа, такие что сумма двух из~них дает
%третье.
%Останется~ли утверждение верным, если $1002$ заменить на~$1001$?

\item
Из~чисел от~$1$ до~$2 n$ выбрано $n + 1$ число.
Докажите, что среди выбранных чисел найдутся два, одно из~которых делится
на~другое.

\end{problems}

\subsubsection*{Разнобой}

\begin{problems}

%\item
%Докажите, что из~$11$ различных бесконечных десятичных дробей можно выбрать две
%такие, которые совпадают в~бесконечном числе разрядов.

\item
По~кругу расставлены $111$ различных натуральных чисел, не~превосходящих $500$.
Могло~ли оказаться, что для каждого из~этих чисел его последняя цифра совпадает
с~последней цифрой суммы всех остальных чисел?
\emph{(2014, 9.1)}

\item
Петя выбрал натуральное число $a > 1$ и~выписал на~доску пятнадцать чисел
$1 + a, 1 + a^2, 1 + a^3, \ldots, 1 + a^{15}$.
Затем он стер несколько чисел так, что каждые два оставшихся числа взаимно
просты.
Какое наибольшее количество чисел могло остаться на~доске?
\emph{(2012, 10.6)}
% ответ: 4

\item
Можно~ли множество всех натуральных чисел разбить на~непересекающиеся конечные
подмножества $A_1, A_2, A_3, \ldots$ так, чтобы при любом натуральном $k$ сумма
всех чисел, входящих в~подмножество $A_k$, равнялась $k+2013$?
\emph{(2013, 10.4, 11.3)}
% ответ: нельзя

\item
Докажите, что существует бесконечно много натуральных чисел, представимых
в~виде суммы четырех кубов и~более, чем миллионом способов.

\iffalse % BEGIN ЖЕСТЬ

% ну ладно, решил. но все равно жесть. -- Ю.Т.
\item
Федя выписывает слева направо бесконечную последовательность ненулевых цифр.
После выписывания каждой цифры он раскладывает на~простые множители
получившееся к~этому моменту число.
Докажите, что однажды этих простых множителей будет больше $100$.

% это жесть, кажется. -- Ю.Т.
\item
Пусть $p = 4 k + 1$~--- простое число, $A$~--- подмножество множества
$\{ 1, 2, \ldots, p - 1 \}$, причем для любых чисел $a, b \in A$ существует
натуральное число $t$, такое что $(a - b - t^2)$ делится на~$p$.
Докажите, что $\abs{A} < \sqrt{p}$.

\fi % END ЖЕСТЬ

\end{problems}

\endgroup % \def\abs

