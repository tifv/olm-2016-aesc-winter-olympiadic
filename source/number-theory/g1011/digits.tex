% $date: 2016-01-10
% $timetable:
%   g1011:
%     2016-01-10:
%       1:

\section*{Теория чисел. Десятичная запись числа}

% $authors:
% - Наталия Евгеньевна Шавгулидзе

\begin{problems}

%\item
%Все натуральные числа выписаны в~ряд.
%Получилась последовательность цифр
%\[
%    1234567891011121314\ldots
%\]
%Какая цифра будет стоять на~$2016$ месте?

\item
Натуральное число~$m$ таково, что сумма цифр числа~$8^m$ равна $8$.
Может~ли $8^m$ оканчиваться на~$6$?

\item
Может~ли натуральное число, сумма цифр которого равна $30$, быть квадратом
какого-нибудь натурального числа?

\item
Известно, что пятизначное число $\overline{abcde}$ делится на~$41$.
Докажите, что, если цифры числа $\overline{abcde}$ циклически переставить,
то~получившееся число тоже делится на~$41$.
(То есть надо доказать, что числа $\overline{bcdea}$, $\overline{cdeab}$,
$\overline{deabc}$, $\overline{eabcd}$ делятся на~$41$.)

\item
Обозначим $S(n)$ сумму цифр числа~$n$.
Пусть $a_n$~--- такая последовательность чисел, что $a_{k+1} = S(a_{k})$ для
любого $k \geq 0$.
Найдите $a_6$, если $a_0 = 2^{1000000}$.

\item
Числа от~$1$ до~$k$ выписаны в~ряд, получилась последовательность цифр.
\\
\subproblem \label{number-theory/digits:problem:5a}
Существуют~ли такие натуральные числа~$k$, что в~этой последовательности каждая
цифра будет встречаться четное число раз?
\\
\subproblem
\ldots существует~ли такие четырехзначные числа~$k$?

%\item
%Можно~ли при каком-то натуральном~$k$ разбить все натуральные числа
%от~$1$ до~$k$ на~две группы и~выписать числа в~каждой группе подряд в~некотором
%порядке так, чтобы получились два одинаковых числа?

\item
Даны натуральные числа $M$ и~$N$, большие десяти, состоящие из~одинакового
количества цифр.
Кроме того, $M = 3 N$.
Чтобы получить число~$M$, надо к~одной из~цифр~$N$ прибавить $2$, а~ко~всем
остальным цифрам по~нечетной цифре.
Какой цифрой могло оканчиваться число~$N$?
Найдите все варианты.

\item
Докажите, что в~любом шестизначном числе можно переставить цифры так, чтобы
сумма первых трех отличалась от~суммы вторых трех меньше, чем на~$10$.

%\item
%На~экране компьютера горит число, которое каждую минуту увеличивается на~$102$.
%Начальное значение числа $123$.
%Глеб имеет возможность в~любой момент изменять порядок цифр числа, находящегося
%на~экране.
%Может~ли он добиться того, чтобы число никогда не~стало четырехзначным?

\item
Натуральные числа $a < b < c$ таковы, что $(b + a)$ делится на~$(b - a)$,
а~$(c + b)$ делится на~$(c - b)$.
Число~$a$ записывается $2011$ цифрами, а~число~$b$~--- $2012$ цифрами.
Сколько цифр в~числе~$c$?

\item
Существует~ли степень двойки, из~которой перестановкой цифр можно получить
другую степень двойки?

\item
Обозначим через $S(x)$ сумму цифр натурального числа~$x$.
Решите уравнение
\[
    x + S(x) + S(S(x)) + S(S(S(x))) = 1993
\, . \]

\item
Докажите, что для любого натурального~$n$ существует делящееся на~него число,
состоящее только ровно из~$n$ единиц и~некоторого количества нулей.

\item
Обозначим через $S(m)$ сумму цифр натурального числа~$m$.
Докажите, что существует бесконечно много натуральных~$n$ таких, что
\[
    S(3^n) \geq S(3^{n+1})
\, . \]

%\item
%Глеб написал на~доске ненулевую цифру и~приписывает к~ней справа по~одной
%ненулевой цифре, пока не~выпишет миллион цифр.
%Докажите, что на~доске не~более $100$~раз был написан точный квадрат.

\end{problems}

