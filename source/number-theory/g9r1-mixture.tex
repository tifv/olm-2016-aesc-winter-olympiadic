% $date: 2016-01-09
% $timetable:
%   g9r1:
%     2016-01-10:
%       2:

\section*{Теория чисел. Разнобой}

% $authors:
% - Юлий Васильевич Тихонов

\begin{problems}

\item
Решите в~натуральных числах уравнение $n^{x} + n^{y} = n^{z}$.
% http://problems.ru/view_problem_details_new.php?id=73766

\end{problems}

\subsection*{Принцип крайнего}

\begin{problems}

\item
На~доску выписаны $2011$ чисел.
Оказалось, что сумма любых трех выписанных чисел также является выписанным
числом.
Какое наименьшее количество нулей может быть среди этих чисел?
%\emph{(2011, 10.6)}

\item
На~доске написано несколько натуральных чисел.
Сумма любых двух из~них~--- натуральная степень двойки.
Какое наибольшее число различных может быть среди чисел на~доске?
% http://problems.ru/view_problem_details_new.php?id=116683

\item
Можно~ли при каком-то натуральном $k$ разбить все натуральные числа
от~$1$ до~$k$ на~две группы и~выписать числа в~каждой группе подряд в~некотором
порядке так, чтобы получились два одинаковых числа?
%\emph{(2010, 9.3, 10.2)}
% наибольшая степень 10

\item
Дано $n$ попарно взаимно простых чисел, больших $1$ и~меньших $(2 n - 1)^{2}$.
Докажите, что среди них обязательно есть простое число.
% http://problems.ru/view_problem_details_new.php?id=34920

\end{problems}


\subsection*{Принцип Дирихле}

\begin{problems}

\item
По~кругу расставлены $111$ различных натуральных чисел, не~превосходящих $500$.
Могло~ли оказаться, что для каждого из~этих чисел его последняя цифра совпадает
с~последней цифрой суммы всех остальных чисел?
%\emph{(2014, 9.1)}

\item
\subproblem
Докажите, что из~любых двенадцати натуральных чисел можно выбрать два,
разность которых делится на~$11$.
\\
\subproblem
В~ряд выписано одиннадцать натуральных чисел.
Докажите, что либо одно из~них делится на~$11$, либо сумма нескольких рядом
стоящих делится на~$11$.

\item
Имеется $2 k + 1$ карточек, занумерованных числами от~$1$ до~$2 k + 1$.
Какое наибольшее число карточек можно выбрать так, чтобы ни~один из~извлеченных
номеров не~был равен сумме двух других извлеченных номеров?

\end{problems}

